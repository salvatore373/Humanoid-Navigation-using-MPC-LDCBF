\section{MPC}
Eugenio: \\
The dynamic model is used to define a MPC problem to compute the optimal stepping positions for save nagivation and locomotion. The LIP-MPC is defined as:
\begin{align*}
    &J^* = \min_{u_{0:N-1}} \sum_{k=1}^{N}[(p_{x_{k}} - g_{x})^2+(p_{y_{k}} - g_{y})^2]
    \\ \text{s. t} \quad
    &x_{k+1} = A_{L}x_{k} + B_{L}u_{k} \qquad
    c_{l} \le c(x_{k}, u_{k}) \le c_{u}
\end{align*}
Minimizing the cost function means that the robot is moving towards the goal position. This problem is subject to the satisfaction of the robot's dynamics and the kinematics and path constraints. The kinematics and path constraints are captured in $c(x_{k}, u_{k})$ and they are often nonlinear. This nonlinearity is related to presence of the heading angle $\theta_{k}$. These constraints are linearized by pre-computing the turning rates $\bar{\omega}_{k}$ for all the horizon length, in order to have them fixed in the MPC calculation.
\\
\\

\section{LIP-MPC: Gait planning with Model Predictive Control}\label{sec:mpc}
Salvatore:\\
The LIP dynamics defined in (\ref{eq:lip_dyanmics}) is used as a model of the process inside a Model Predictive Control (MPC) scheme. The MPC controller uses that model to predict the future output along a \textit{prediction horizon} $N$, namely the predefined number of time steps to look out in the future. Based on those forecasts and the provided constraints, the MPC computes the sequence of control actions that optimize a given cost function during the prediction horizon. Then, only the first input of that sequence is taken and provided to the process. The real output will be used at the next time step to compute the new predictions and control actions.\\
 In our case, the LIP-MPC is used to respond instantaneously to the changes in the humanoid's state, while providing optimal stepping positions for stable locomotion and safe navigation. It is formulated as follows:

\begin{align}
    J^* &= \min_{\mathbf{u}_{0:N-1}} \sum_{k=1}^{N} q(\mathbf{x}_k) \\
    \text{s.t.} \quad
    &\mathbf{x}_k \in X, \quad k \in [1, N] \notag \\
    &\mathbf{u}_k \in U, \quad k \in [0, N-1] \notag \\
    &\mathbf{x}_{k+1} = \mathbf{A_L} \mathbf{x}_k + \mathbf{B_L} \mathbf{u}_k, \quad k \in [0, N-1] \notag \\
    &\mathbf{c}_l \leq \mathbf{c}(\mathbf{x}_k, \mathbf{u}_k) \leq \mathbf{c}_u, \quad k \in [0, N-1], \notag
    \label{eq:lip-mpc}
\end{align}

where $q(\mathbf{x}_k)$ is the cost function to minimize along the prediction horizon. It drives the humanoid toward the goal by minimizing the distance between its current position and the target position. It is defined as:

$$
q(\mathbf{x}_k) = \left( p_{x_k} - g_x \right)^2 + \left( p_{y_k} - g_y \right)^2 \qquad \forall k \in \left[1, N\right],
$$

where the goal position $(g_x, g_y)$ is expressed in the humanoid's local RF. The LIP dynamics is included in the MPC definition to specify how the future states are predicted. Whereas, all the constraints that the optimization problem is subject to are captured by $\mathbf{c}(\mathbf{x}_k, \mathbf{u}_k)$. In order to reduce the computational load, they are all expressed linearly, and they include the walking velocities, leg reachability, and maneuverability constraints, and the linear control barrier function, which will be described in details in the following sections.

\subsection{Heading Angle Preprocessing}
Many of the constraints imposed in the MPC make use of $\theta$ in a non-linear way: for example, the kinematic constraints often show sinusoidal terms having $\theta$ as argument. This hinders real-time computation. The solution proposed by Peng et al. in \cite{peng_main_paper} consists in using precomputed values for $\theta$ and $\omega$, and it is often sufficient to linearize some constraints.\\
The values of $\theta$ and $\omega$ are not determined by the optimization of the previously defined cost function performed by the MPC. Instead, their values throughout the prediction horizon are computed at the beginning of each time step with the following formulae:

\begin{gather}
\omega_k = \max \left\{ \min \left\{ \operatorname{atan2}(g_y - p_y,\, g_x - p_x) - \theta,\, \quad \omega_{max} \right\} ,\quad \omega_{min} \right\} \qquad \forall k \in \left[0, N-1 \right], \notag \\
\theta_0 = \theta, \qquad \theta_k = \theta_{k-1} + \omega_k\;T \qquad \forall k \in \left[1, N\right], \notag
\end{gather}

\todo[inline]{change theta precomputation and explain the new one}

where $p_x, p_y$ and $\theta$ are the position and orientation of the humanoid at the start of the simulation time step, $T$ is the sampling time, and the goal position is expressed in local coordinates. $\omega_{min}$ and $\omega_{max}$ are the bounds on the robot turning rate: they must be added due to the actuation limits and to avoid sharp turns, which would threaten the stability of the humanoid. With the specified formulae, $\theta$ rotates with a velocity $\omega$ until the robot points to the goal.

\subsection{Kinematic Constraints}
The stepping positions provided by the MPC solution must comply with specific kinematic constraints in order to be physically feasible. In this work, Peng et al. decided to enforce the walking velocities, leg reachability, and maneuverability constraints defined as follows.
\todo[inline]{Better explain all the constraints}

\subsubsection{Walking Velocities Constraint}

\todo[inline]{Change this constraint to adhere to Michele\&Andrew's suggestions.}

\[
\begin{pmatrix}
v_{x_{\min}} \\
v_{y_{\min}}
\end{pmatrix}
\le
\begin{pmatrix}
\cos \theta_k & \sin \theta_k \\
-\sin \theta_k & \cos \theta_k
\end{pmatrix}
\begin{pmatrix}
v_{x_{k+1}} \\
s_v v_{y_{k+1}}
\end{pmatrix}
\le
\begin{pmatrix}
v_{x_{\max}} \\
v_{y_{\max}}
\end{pmatrix}
\qquad \forall k \in \left[ 0,\, N-1\right].
\]\\


The pre-multiplying matrix rotates the velocities vector by an angle $-\theta_k$ around the $z$-axis: namely, it takes the velocity vector $\mathbf{v}_{k+1}$ back to the orientation of $RF_{k}$. In the vector resulting from the transformation, the first component is the forward velocity, while the second is the lateral velocity, both in $RF_k$. At this point, the constraint imposes that the velocity is within the lateral and longitudinal limits, $v_{y_{\{\min,\, \max\}}}$ and $v_{x_{\{\min,\, \max\}}}$ respectively.\\
$s_v$ is $1$ if stance foot is the right one, $-1$ otherwise. It is meant to limit the lateral velocity in such a way that, at the end of each step, the stance foot is on the opposite side but the humanoid does not lose balance.
%% TODO: decide whether the text below should be removed
For instance, \todo{decide whether this text should be removed} if the right foot is the stance, since the body tends to fall on the left side, we want the lateral velocity not to increase too much toward the direction of the positive $y$. Therefore, we set $s_v=1$, pushing the MPC to find a solution which is closer to $v_{y_{\min}}$.

\subsubsection{Leg Reachability Constraint}
\[
\begin{pmatrix}
-l_{\max} \\
-l_{\max}
\end{pmatrix}
\le
\begin{pmatrix}
\cos \theta_k & \sin \theta_k \\
-\sin \theta_k & \cos \theta_k
\end{pmatrix}
\begin{pmatrix}
p_{x_k} \\
p_{y_k}
\end{pmatrix}
\le
\begin{pmatrix}
l_{\max} \\
l_{\max}
\end{pmatrix}
\qquad \forall k \in \left[0,\, N-1 \right].
\]

As stated before, the matrix in this constraint transforms the orientation of $(p_{x_k},\,p_{y_k})$ to the one of $RF_{k-1}$. It prevents the inputs computed at time $k$ from moving the CoM too far away from $(p_{x_{k-1}},\,p_{y_{k-1}})$. Indeed, the distance between the previous and the new position cannot be farther than $l_{\max}$, which is the maximum reachable distance of the swing foot in both directions on the ground.


\subsubsection{Maneuverability Constraint}
\[
\begin{pmatrix}
\cos \theta_k & \sin \theta_k
\end{pmatrix}
\begin{pmatrix}
v_{x_k} \\
v_{y_k}
\end{pmatrix}
\leq v_{x_{\max}} - \frac{\alpha}{\pi} |\omega_k|
\qquad \forall k \in \left[0,\, N-1 \right].
\]

This constraint is meant to reduce the humanoid's walking speed while it is turning. For this reason, it combines the longitudinal and angular velocities. It ensures that the component of the CoM velocity in the robot's heading direction (given by the vectors projection on the left side) is lower than a quantity that depends on the turning rate and some constant values. $\alpha$ is a simulation parameter.

\subsection{Control Barrier Functions}

\subsubsection{Definitions}
Considering a continuous and differentiable function $h: \mathbb{R}^n \rightarrow \mathbb{R}$, we can define the \textit{safety set} $S$ and its boundary $\partial S$ as:
\begin{align*}
S \coloneqq \left\{ \mathbf{x} \in X \mid  h(\mathbf{x}) \geq 0\right\}, \notag \\
\partial S \coloneqq \left\{ \mathbf{x} \in X \mid  h(\mathbf{x}) = 0\right\}. \notag  \\
\end{align*}

They are the region and its perimeter where the robot can freely move without colliding with obstacles.
According to \cite{zeng2021safetycriticalmodelpredictivecontrol}, % formula 5 + remark 1
assume that there exists a class $\mathcal{K}$ function $\gamma$ such that
$$
0 < \gamma(h(\mathbf{x})) \leq h(\mathbf{x}),
$$
and the following holds:
\begin{gather}
\Delta h(\mathbf{x}_k,\, \mathbf{u}_k) \coloneqq h(\mathbf{x}_{k+1}) - h(\mathbf{x}_{k}), \notag \\
\forall \mathbf{x}_k \in S \qquad \exists \mathbf{u}_k \; \text{s.t.} \; \Delta h(\mathbf{x}_k,\, \mathbf{u}_k) \geq -\gamma(h(\mathbf{x}_k)). \label{eq:dcbf_def_constr}
\end{gather}

Then, $h(\cdot)$ is a discrete control barrier function (DCBF). We can also take $\gamma$ as a scalar such that $0 < \gamma \leq 1$, and (\ref{eq:dcbf_def_constr}) becomes:

\begin{gather*}
    \forall \mathbf{x}_k \in S \qquad \exists \mathbf{u}_k \; \text{s.t.} \; \Delta h(\mathbf{x}_k,\, \mathbf{u}_k) \geq -\gamma * h(\mathbf{x}_k). 
\end{gather*}

%%% TODO: not sure about it
It means that, if the state starts from the safety set $S$, there exists an input that, when applied, produces a state which will sill be in $S$: namely, $S$ is invariant. \todo{Not sure about it}

\subsubsection{Linear Discrete Control Barrier Functions (LDCBF)}
Assume that for each obstacle there is a function of the robot's configuration such that $F(\mathbf{x})=0$ when the robot touches the boundary of the obstacle, $F(\mathbf{x})>0$ when it does not collide with the obstacle, and $F(\mathbf{x})<0$ if it is inside the obstacle. It is convenient to choose $h(\cdot)=F(\cdot)$. This is the case in our project, where the DCBF is a function of $\vec{x}$, the Cartesian position of the CoM: therefore, $h(\cdot)\colon \mathbb{R}^2 \rightarrow \mathbb{R}$. Moreover, we assume that either the obstacle associated with $F(\cdot)$ is convex or $F(\cdot)$ describes the convex shape that encloses the non-convex obstacle.
However, if $F(\cdot)$ is a non-linear function, we obtain a non-linear DCBF constraint, which should be avoided for the reasons exposed before.\\
Peng et al. in \cite{peng_main_paper} linearized the DCBF by approximating the safe region. The plane is partitioned in two halves: one containing the obstacle and the other containing the robot (as shown in Fig. \ref{fig:example_LDCBF}).
The safe region is the latter, and it is defined as:

\begin{equation} \notag
h\left(\vec{x}\right) = \eta^T \left(\vec{x} - c\right) \geq 0,
\end{equation}

where $c \in \mathbb{R}^2$ is the point on the obstacle's edge that is closest to the CoM position $\vec{x}$, while $\eta \in \mathbb{R}^2$ is the normal vector that connects $\vec{x}$ to the boundary. Hence, $\eta$ is also the vector orthogonal to the line that defines the half-planes, pointing toward the safe region. All of these components are shown in Figure (\ref{fig:example_LDCBF}).\\

\begin{figure}[h]
    \centering
    \includegraphics[width=0.75\linewidth]{figures//MPC/example_lcbf.pdf}
    \caption{In this plot, the CoM is represented as a green point, and the obstacle is a blue rectangle. The point on the obstacle's boundary that is closest to $\vec{x}$ is the red one, which represents $c$. The red vector connecting $c$ to $\vec{x}$ is $\eta$. The safe region defined by this LDCBF is the gray area. In that half-plane, the humanoid's CoM is free to move without hitting any obstacle.}
    \label{fig:example_LDCBF}
\end{figure}

If multiple obstacles are in the environment, the safe region is the intersection of the regions produced by each LDCBF.\\
By enforcing that the states $\mathbf{x}_{1...N+1}$ produced by the optimal inputs $\mathbf{u}_{0...N}$ computed by the MPC satisfy $h(\vec{x})>0$, we guarantee that, during the motion, the humanoid's CoM does never collide with the obstacles.\\

Vectors $\eta$ and $c$ are computed as follows. Indicating with $A$ and $B$ the endpoints of the edge of the obstacle that is closest to $\vec{x}$ we have:
\begin{gather*}
    \overrightarrow{AX} \coloneqq \vec{x} - A, \qquad \overrightarrow{AB} \coloneqq B - A, \\
    t \coloneqq \frac{\overrightarrow{AX} \cdot \overrightarrow{AB}}{\left\lVert \overrightarrow{AB} \right\rVert ^2}, \qquad \tilde{t} = \max \left\{0,\; \min \left\{ 1,\; t\right\}\right\}, \\
    c \coloneqq A\, +\, \tilde{t}*\overrightarrow{AB}, \\
    \eta \coloneqq \left(-1\right)^{\gamma} * \left( \vec{x} - c \right),
\end{gather*}

where $\gamma=1$ if $\vec{x}$ is inside the obstacle, otherwise it is $0$.
Due to the presence of $\min$ and $\max$, the specified expressions of $c$ and $\eta$ are non-linear. However, the constraint enforced on the MPC solution at simulation time $k$ is based on the values of $c$ and $\eta$ computed at the beginning of the time step. Therefore, they are included in the constraint as constants, and the LDCBF becomes a linear combination of $\vec{x}$.

\subsubsection{Our variation of the LDCBF}
From Figure (\ref{fig:example_LDCBF}) it is evident that a such defined LDCBF constraint allows the humanoid to get very close to the obstacle's boundary. And, even though the CoM will not collide with it, the footstep positions provided by the MPC will likely overlap with the obstacle. Therefore, we developed a variant of the LDCBF proposed by Peng et al., defined as follows:

\begin{equation} \label{eq:ldcbf_variant}
h\left(\vec{x}\right) = \eta^T \left(\vec{x} - c\right) - \delta \geq 0,
\end{equation}

where $c$ and $\eta$ are the previously defined vectors, while $\delta$ is the distance that the CoM must keep from the $c$ (i.e., from the obstacle's boundary) to satisfy the constraint. The result is shown in Figure \ref{fig:example_LDCBF_variant}\\
Forcing the control barrier function not to take values below a threshold allows us to take into account the footstep positions, and to generate trajectories that are not only feasible for the CoM, but for the motion of the whole humanoid.

\begin{figure}[h]
    \centering
    \includegraphics[width=0.75\linewidth]{figures//MPC/example_ldcbf_variant.pdf}
    \caption{This plot shows the same environment of Figure \ref{fig:example_LDCBF}, but here expression (\ref{eq:ldcbf_variant}) with $\delta > 0$ was used as LDCBF. It is clear that the safe zone is not adjacent to the obstacle, and $\vec{x}$ maintains a distance $\delta$ from the obstacle.}
    \label{fig:example_LDCBF_variant}
\end{figure}